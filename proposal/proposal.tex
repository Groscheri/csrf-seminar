\documentclass[a4paper,11pt]{article}
\usepackage[utf8]{inputenc}
\usepackage[english]{babel}
\usepackage[font=small,labelfont=bf, textfont=it]{caption}
\usepackage[bottom=2cm, left=3.7cm, right=3.7cm]{geometry} % to change padding
\usepackage{verbatim}
\usepackage{subcaption} % for multi figure
\usepackage{enumitem} % -- label item
\usepackage{tabularx}
\usepackage{color}
\usepackage[usenames, dvipsnames]{xcolor} % color
\usepackage[framemethod=TikZ]{mdframed} % box
\usepackage{listings} % code
\usepackage{minted} % code
\usepackage{amsmath} % align* maths
\usepackage{wrapfig}
\usepackage[bookmarks, hidelinks]{hyperref} % clickable links
\usepackage{graphicx} % includegraphics
\usepackage[section]{placeins} % float inside section
\usepackage{indentfirst}

\setlength\parindent{0pt}
\setlength{\parskip}{1em}

% TODO Title of the seminar CSRF Attack: even Google was vulnerable
\title{Internet Security \& Privacy\\Seminar -- Proposal\\\vspace{10pt}\textbf{CSRF attack: even Google was vulnerable.}}
\author{David KUFA \& Quentin LEMAIRE\\Group 26}

\newcommand{\csrf}{\textit{Cross-Site Request Forgery}}

\begin{document}

  \maketitle % build title

  \begin{abstract}
  \csrf{} (CSRF) -- often spelled ``sea surf'' -- is a well-known web attack which has 
  been discovered in 2001. The \textit{Open Web Application Security Project} (OWASP \cite{owasp}) 
  ranked CSRF as the 8th\footnote{In 2007 OWASP's top ten, CSRF was at the 5th position 
  (\url{https://www.owasp.org/index.php/Top_10_2007}).} vulnerabily in the top 10 of the 
  most critical web application security risks in 2013~\cite{owasp_top_ten}.
  
  CSRF attack consists in creating (forging) fake HTTP or HTTPS\footnote{TLS encrypts 
  information between the client (browser) and the server in order to prevent sniffing 
  of untrusted networks and man in the middle attacks but it doesn't protect from 
  legitimate requests.} requests on the user's behalf. It utilizes the lack of knowledge 
  of the victim to build the request and get information with their credentials (as if 
  the user really wanted to execute this request). In order to succeed, the victim must 
  be connected (authenticated) to the service (website) where there is the vulnerability. 
  Then, an attacker will have to fool the victim in order to build the fake request (with 
  social engineering for instance).
  
  During this seminar, we want to get a better understanding of the security breaches 
  involved in CSRF attack. The most interesting part consists in the comprehension of 
  the surface of attack, how this attack can be done and where does it come from ? It 
  is also important to understand what kind of information an attacker could steal or 
  affect on the user's behalf thanks to this attack. Furthermore, it is relevant to 
  know how to detect the vulnerability and how to protect web servers from this attack.
  
  Lots of companies were affected in the past and we will focus our attention on Google 
  in particular with 2 different vulnerabilities discovered in 2007 concerning Gmail
  \footnote{\url{https://mail.google.com/}} email service.
  \end{abstract}


  \section{Meta-information}
  
  \begin{description}
   \item[Group number] 26
   \item[Group members] \textsc{David KUFA} \& \textsc{Quentin LEMAIRE}
   \item[Title] CSRF attack: even Google was vulnerable.
   \item[Schedule conflicts (Dec 15 \& 16)] None
  \end{description}

  \section{Outlines}
  
  % TODO section headlines
  % More concrete description of topic proposal wrote on the wiki page
  % Final report must ideally be extended from proposal
  
  % What will you do?
  % Which tools will you use?
  % What's the goal (tutorial, security recommendations, a design description, implementation, ...)?
  % What are the limitations? What will you NOT do?
  
  The goal of this seminar is to get deeper understanding on the CSRF attack. We believe 
  that a theoretical part associated with practical examples is mandatory in order to fully understand 
  where this vulnerability comes from and how to prevent it. That's why we will focus our attention 
  on the implementation of a small web server delivering vulnerable and none vulnerable 
  services (from CSRF attack). This practical part must be short enough in order to, in a second part, focus our 
  attention on 2 Google attacks due to CSRF vulnerability. We will cover the specificities 
  of both attacks and we will see the differences and the impacts of those two security 
  breaches.
  
  \subsection{Seminar plan}
  % plan
  
  \begin{enumerate}
   \item CSRF: how it works ? Implementation of a vulnerable web service.
   \item CSRF: how to prevent it ? Implementation of a protected web service with different methods of protection.
   \item Some stories about CSRF: 2 Google vulnerabilities in 2007.
   \begin{itemize}
    \item Contact list spoofing: it was possible to retrieve all the contact list of an user.
    \item Email filter hijacking: it was possible to forward all email of an user to another selected email address. This attack had a great impact on Gmail users trust.
   \end{itemize}
  \end{enumerate}
  
  \subsection{Practical part}
  
  We are aware that it will be difficult to make a demonstration during the oral presentation 
  because we don't have too much time for this. That's why we won't show our practical part 
  during the oral presentation but we may use this part as a common thread for our explanations. 
  
  Technically speaking, we will implement a small platform using either \textit{Apache 2.4}\footnote{\url{https://httpd.apache.org/}} or \textit{nginx}\footnote{\url{http://nginx.org/}} as a web 
  server. We own a small \textit{Raspberry Pi}\footnote{\url{https://www.raspberrypi.org/}} in which we can develop this platform. It will contains several 
  things:
  \begin{itemize}
   \item An authentication system, which is part of the attack, an attack wants to forge request on the behalf of an user, and he or she has to be connected.
   \item A vulnerable service: a GET or POST service which updates data about the user (password for instance).
   \item Some protected services with different protection methods:
   \begin{itemize}
    \item CSRF tokens known from the server and the client ;
    \item Signed tokens with a private secret (only known by the server), this is a simple way to avoid saving and storing tokens but this method has several drawbacks like ``golden'' token (if the token is stolen, it can be reused on the user's behalf). This protection method will use what we've learnt from the symmetric encryption and hashing lectures of this course ;
    \item \textit{HTTP-REFERER} check but since this value comes from the client (which has been tricked), it can't be trusted.
   \end{itemize}
  \end{itemize}
  
  According to the programming language, we don't have any preferences. Python seems to be very easy to 
  compute with libraries like \textit{Flask}\footnote{\url{http://flask.pocoo.org/}} or 
  \textit{Django}\footnote{\url{https://www.djangoproject.com/}} (but we won't use Django because it 
  already provides tools against CSRF attack). We can also build a small website with PHP from 
  scratch. This language provides lots of database integration tools\footnote{PHP provides classes as \textit{PDO} (\url{http://php.net/manual/en/book.pdo.php}) 
  which protects from SQL injections thanks to prepared statements but this is not the point of this seminar.} 
  and session management.
  
  \subsection{Limitations}

  In this seminar, we won't study browser mechanisms which allow or prevent cross-origin 
  request (XHR request from JavaScript for instance). Moreover, we won't go deep in HTTP 
  protocol~\cite{rfc2616} and we will focus our attention on GET and POST methods 
  only~\cite[5.1.1]{rfc2616}. Finally, we won't cover all strategies (implemented by lots 
  of framework) to prevent CSRF attack, but we will try to be as exhaustive as possible.
  
  \section{Conclusion}
  
  % TODO
  
% bibliography
\bibliographystyle{plainurl}
\bibliography{references}

\end{document}
