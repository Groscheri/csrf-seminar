\documentclass[a4paper,11pt]{article}
\usepackage[utf8]{inputenc}
\usepackage[english]{babel}
\usepackage[font=small,labelfont=bf, textfont=it]{caption}
\usepackage[bottom=2cm, left=3.7cm, right=3.7cm]{geometry} % to change padding
\usepackage{verbatim}
\usepackage{subcaption} % for multi figure
\usepackage{enumitem} % -- label item
\usepackage{tabularx}
\usepackage{color}
\usepackage[usenames, dvipsnames]{xcolor} % color
\usepackage[framemethod=TikZ]{mdframed} % box
\usepackage{listings} % code
\usepackage{minted} % code
\usepackage{amsmath} % align* maths
\usepackage{wrapfig}
\usepackage[bookmarks, hidelinks]{hyperref} % clickable links
\usepackage{graphicx} % includegraphics
\usepackage[section]{placeins} % float inside section
\usepackage{indentfirst}

\setlength\parindent{0pt}
\setlength{\parskip}{1em}

% TODO Title of the seminar CSRF Attack: even Google was vulnerable
\title{Internet Security \& Privacy\\Seminar -- Proposal\\\vspace{10pt}\textbf{CSRF attack: even Google was vulnerable.}}
\author{David KUFA \& Quentin LEMAIRE\\Group 26}

\newcommand{\csrf}{\textit{Cross-Site Request Forgery}}

\begin{document}

  \maketitle % build title

  \begin{abstract}
  \csrf{} (CSRF) -- often spelled ``sea surf'' -- is a well-known web attack which has 
  been discovered in 2001. The \textit{Open Web Application Security Project} (OWASP \cite{owasp}) 
  ranked CSRF as the 8th\footnote{In 2007 OWASP's top ten, CSRF was at the 5th position 
  (\url{https://www.owasp.org/index.php/Top_10_2007}).} vulnerabily in the top 10 of the 
  most critical web application security risks in 2013~\cite{owasp_top_ten}.
  
  CSRF attack consists in creating (forging) fake HTTP or HTTPS\footnote{TLS encrypts 
  information between the client (browser) and the server in order to prevent sniffing 
  of untrusted networks and man in the middle attacks but it doesn't protect from 
  legitimate requests.} requests on the user's behalf. It utilizes the lack of knowledge 
  of the victim to build the request and get information with their credentials (as if 
  the user really wanted to execute this request). In order to succeed, the victim must 
  be connected (authenticated) to the service (website) where there is the vulnerability. 
  Then, an attacker will have to fool the victim in order to build the fake request (with 
  social engineering for instance).
  
  During this seminar, we want to get a better understanding of the security breaches 
  involved in CSRF attack. The most interesting part consists in the comprehension of 
  the surface of attack, how this attack can be done and where does it come from ? It 
  is also important to understand what kind of information an attacker could steal or 
  affect on the user's behalf thanks to this attack. Furthermore, it is relevant to 
  know how to detect the vulnerability and how to protect web servers from this attack.
  
  Lots of companies were affected in the past and we will focus our attention on Google 
  in particular with 2 different vulnerabilities discovered in 2007 concerning Gmail
  \footnote{https://mail.google.com/} email service.
  \end{abstract}


  \section{Meta-information}
  
  \begin{description}
   \item[Group number] 26
   \item[Group members] \textsc{David KUFA} \& \textsc{Quentin LEMAIRE}
   \item[Title] CSRF attack: even Google was vulnerable.
   \item[Schedule conflicts (Dec 15 \& 16)] None
  \end{description}

  \section{Outlines}
  
  % TODO section headlines
  % More concrete description of topic proposal wrote on the wiki page
  % Final report must ideally be extended from proposal
  
  % What will you do?
  % Which tools will you use?
  % What's the goal (tutorial, security recommendations, a design description, implementation, ...)?
  % What are the limitations? What will you NOT do?
  
  The goal of this seminar is to get deeper understanding on the CSRF attack. We believe 
  that a theoretical part associated with practical examples is mandatory in order to fully understand 
  where this vulnerability comes from and how to prevent it. That's why we will focus our attention 
  on the implementation of a small web server delivering vulnerable and none vulnerable 
  services (from CSRF attack). This practical part must be short enough in order to, in a second part, focus our 
  attention on 2 Google attacks due to CSRF vulnerability. We will cover the specificities 
  of both attacks and we will see the differences and the impacts of those two security 
  breaches.
  
  \subsection{Seminar plan}
  % plan
  
  \begin{enumerate}
   \item CSRF: how it works ?
   \item CSRF: how to prevent it ?
   \item Some stories about CSRF
   \begin{itemize}
    \item Google: 2 vulnerabilities in 2007 (contact list spoofing \& email filter hijacking)
    \item Other important attacks [TODO] % TODO
   \end{itemize}
  \end{enumerate}
  
  \subsection{Practical part}
  
  % TODO
  
  \subsection{Limitations}

  In this seminar, we won't study browser mechanisms which allow or prevent cross-origin 
  request (XHR request from JavaScript for instance). Moreover, we won't go deep in HTTP 
  protocol~\cite{rfc2616} and we will focus our attention on GET and POST methods only~\cite[5.1.1]{rfc2616}.
  % TODO to complete
  
% bibliography
\bibliographystyle{plainurl}
\bibliography{references}

\end{document}
