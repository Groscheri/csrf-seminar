\documentclass[a4paper,11pt]{article}
\usepackage[utf8]{inputenc}
\usepackage[english]{babel}
\usepackage[font=small,labelfont=bf, textfont=it]{caption}
\usepackage[bottom=2cm, left=3.7cm, right=3.7cm]{geometry} % to change padding
\usepackage{verbatim}
\usepackage{subcaption} % for multi figure
\usepackage{enumitem} % -- label item
\usepackage{tabularx}
\usepackage{color}
\usepackage[usenames, dvipsnames]{xcolor} % color
\usepackage[framemethod=TikZ]{mdframed} % box
\usepackage{listings} % code
\usepackage{minted} % code
\usepackage{amsmath} % align* maths
\usepackage{wrapfig}
\usepackage[bookmarks, hidelinks]{hyperref} % clickable links
\usepackage{graphicx} % includegraphics
\usepackage[section]{placeins} % float inside section
\usepackage{indentfirst}

\setlength\parindent{0pt}
\setlength{\parskip}{1em}

% TODO Title of the seminar CSRF Attack: even Google was vulnerable
\title{Internet Security \& Privacy\\Seminar -- Proposal\\\vspace{10pt}\textbf{CSRF attack: even Google was vulnerable.}}
\author{David KUFA \& Quentin LEMAIRE\\Group 26}

\newcommand{\csrf}{\textit{Cross-Site Request Forgery}}

\begin{document}

  \maketitle % build title

  \begin{abstract}
  % TODO abstract/resume of the seminar
  \csrf{} (CSRF) -- often spelled ``sea surf'' -- is a well-known web attack which has 
  been discovered in 2001. The \textit{Open Web Application Security Project} (OWASP \cite{owasp}) 
  ranked CSRF as the 8th vulnerabily in the top 10 of the most critical web application 
  security risks in 2013~\cite{owasp_top_ten}.
  % TODO to complete

  \end{abstract}
  
% What will you do?
% Which tools will you use?
% What's the goal (tutorial, security recommendations, a design description, implementation, ...)?
% What are the limitations? What will you NOT do?


  \section{Meta-information}

  % TODO schedule conflicts (Dec 15 & 16)
  % TODO title

  \section{Outlines}
  
  % TODO section headlines
  % More concrete description of topic proposal wrote on the wiki page
  % Final report must ideally be extended from proposal
  
  The goal of this seminar is to get further understanding on the CSRF attack. We believe 
  that a theoretical part with example (practical) is mandatory in order to fully understand 
  how this vulnerability exists and how to prevent it. That's why we will focus our attention 
  on the implementation of a small web server delivering vulnerable and none vulnerable 
  services. This practical part must be short enough in order to, in a second part, focus our 
  attention on 2 Google attacks due to CSRF vulnerability. We will cover the specificities 
  of both attacks and we will see the differences and the impacts of those two security 
  breaches.
  
  \subsection{Seminar plan}
  % plan
  
  \begin{enumerate}
   \item CSRF: how it works ?
   \item CSRF: how to prevent it ?
   \item Some stories about CSRF
   \begin{itemize}
    \item Google: 2 vulnerabilities in 2007 (contact list spoofing \& email filter hijacking)
    \item Other important attacks [TODO] % TODO
   \end{itemize}
  \end{enumerate}
  
  \subsection{Practical part}
  
  \subsection{Limitations}

  
% bibliography
\bibliographystyle{plain}
\bibliography{references}

\end{document}
