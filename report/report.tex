\documentclass[a4paper,11pt]{report}
\usepackage[utf8]{inputenc}
\usepackage[english]{babel}
\usepackage[font=small,labelfont=bf, textfont=it]{caption}
%\usepackage[bottom=2cm, left=2.5cm, right=2.5cm]{geometry} % to change padding
\usepackage{verbatim}
\usepackage{subcaption} % for multi figure
\usepackage{enumitem} % -- label item
\usepackage{tabularx}
\usepackage{color}
\usepackage[usenames, dvipsnames]{xcolor} % color
\usepackage[framemethod=TikZ]{mdframed} % box
\usepackage{listings} % code
%\usepackage{minted} % code
%\usepackage{amsmath} % align* maths
\usepackage{wrapfig}
\usepackage[bookmarks, hidelinks]{hyperref} % clickable links
\usepackage{graphicx} % includegraphics
\usepackage[section]{placeins} % float inside section
\usepackage{indentfirst}

\setlength\parindent{0pt}
\setlength{\parskip}{1em}

\renewcommand\thesection{\arabic{section}} % start section from 1
\setcounter{tocdepth}{2} % display subsection
\setcounter{secnumdepth}{3} % number subsubsection

% \title{Internet Security \& Privacy\\Seminar -- Report\\\vspace{10pt}\textbf{CSRF attack: even Google was vulnerable.}}
% \author{Quentin Lemaire \& David Kufa\\Group 26}

\newcommand{\csrf}{\textit{Cross-Site Request Forgery}}


\begin{document}

  % first page
 \begin{titlepage}
  \centering
  \vspace*{\stretch{1}}
  \vfill
    {\bfseries\Large{
	Internet Security \& Privacy\\
	Seminar -- Report}
    }    
  \vfill
  \vfill
    \Huge{\textsc{CSRF attack: even Google was vulnerable.}}
  \vfill
      \Large{\textsc{David Kufa} \& \textsc{Quentin Lemaire}}
    \\
  \vspace{0.4cm}
    Group 26
  \vfill
  \vfill
    % \includegraphics[width=0.22\textwidth]{kth.jpg}
    KTH Royal Institute of Technology
  \vfill
    \today
  \vspace*{\stretch{1}}
\end{titlepage}

% abstract
\begin{abstract}
  \csrf{} (CSRF) -- often spelled ``sea surf'' -- is a well-known web attack which has 
  been discovered in 2001. The \textit{Open Web Application Security Project} (OWASP \cite{owasp}) 
  ranked CSRF as the 8th\footnote{In 2007 \& 2010 OWASP's tops ten, CSRF was at the 5th position 
  (\url{https://www.owasp.org/index.php/Top_10_2007} \& \url{https://www.owasp.org/index.php/Top_10_2010}).} 
  vulnerabily in the top 10 of the most critical web application security risks in 2013~\cite{owasp_top_ten}.
  
  CSRF attack consists in creating (forging) fake HTTP or HTTPS\footnote{TLS encrypts 
  information between the client (browser) and the server in order to prevent sniffing 
  in untrusted networks and man in the middle attacks but it doesn't protect from 
  legitimate requests.} requests on the user's behalf. It utilizes the lack of knowledge 
  of the victim to build the request and get information with their credentials (as if 
  the user really wanted to execute this request). In order to succeed, the victim must 
  be connected (authenticated) to the service (website) where there is the vulnerability. 
  Then, an attacker will have to fool the victim in order to build the fake request (with 
  social engineering for instance).
  
  During this seminar, we wanted to get a better understanding of the security breaches 
  involved in CSRF attack. The most interesting part consisted in the comprehension of 
  the surface of attack, how this attack can be done and where does it come from. It 
  was also important to understand what kind of information an attacker could steal or 
  affect on the user's behalf thanks to this attack. Furthermore, it was relevant to study 
  different ways of detecting the vulnerability and how to protect web servers from this 
  attack.
  
  In a second part, we focused our attention on concrete applications of CSRF vulnerabilities. 
  Lots of companies were affected in the past and we decided to deal with Google well-known 
  stories about CSRF. Indeed, 2 different vulnerabilities were discovered in 2007 concerning 
  Gmail \footnote{\url{https://mail.google.com/}} email service.
  
  Finally, we implemented a small webapp with different services, either vulnerable or protected 
  from CSRF attack. This webapp associated with an attacker website, both developped from scratch, 
  outline different practical examples of the attack and different ways to prevent it.
  \end{abstract}
  
%   \begin{enumerate}
%    \item CSRF: how it works ? Implementation of a vulnerable web service.
%    \item CSRF: how to prevent it ? Implementation of a protected web service with different methods of protection.
%    \item Some stories about CSRF: 2 Google vulnerabilities in 2007.
%    \begin{itemize}
%     \item Contact list spoofing: it was possible to retrieve all the contact list of an user~\cite{gmail_contact_list_csrf, gmail_contact_list_csrf2}.
%     \item Email filter hijacking: it was possible to forward all email of an user to another selected email address~\cite{gmail_hijack_csrf, gmail_hijack_csrf2}. This attack had a great impact on Gmail users trust.
%    \end{itemize}
%   \end{enumerate}

\tableofcontents{} % toc
\clearpage % leave a page
\setcounter{page}{1} % init counter page

  % TODO part1: overview of CSRF
  % TODO     * Description & Identification
  % TODO     * Exploitation
  % TODO     * Recommandations: how to prevent CSRF ?
  
  % TODO part2: Google stories: how ? impact ?
  % TODO     * Contact list spoofing
  % TODO     * Email filter hijacking

  \section{Introduction}
  
  This report is split in two different parts. First, it deals with \csrf{} (CSRF) attack in 
  a theoretical and practical way. Academic examples and explanations describe and explore 
  different aspects and subtilities of CSRF. The second part of this report is about concrete 
  uses (in ``real life'') of CSRF attack with 2 use cases about Google Gmail service.

  \section{Overview of CSRF}
  
  For each kind of WEB vulnerabilities, it is important to know how to detect them and 
  how to prevent them. We will explain in more details what is CSRF attack, how to 
  identify and exploit CSRF vulnerable services and finally how to protect these services. 
  Every explanations will be followed with example from a web application we developped. 
  This web application provides different services in order to manage and store a list of 
  interests (hobbies). More explanations from the technical documentation can be found 
  in appendix~\ref{app:practical_documentation}.
  
  \subsection{Description \& Identification}
  
  \subsection{Exploitation}
  
  \subsection{How to prevent it ?}
  
  \subsubsection{Tokens}
  
  \paragraph{Session token} % each session has an associated token (this token can change for each action done)
  \paragraph{Signed token} % show code as example (using HMAC)
  
  \subsubsection{Challenges}
  
  \paragraph{Double authentication} % retype password
  \paragraph{CAPTCHA} % find a text inside a picture (similar to token)
  
  \subsubsection{Other protections}
  
  Recommandations described above are not exhaustives. It is possible to combine them in order to increase 
  the security of the application against CSRF attack.
  
  
%   We are aware that it will be difficult to make a demonstration during the oral presentation 
%   because we don't have too much time for this. That's why we won't show our practical part 
%   during the oral presentation but we may use this part as a common thread for our explanations. 
%   
%   Technically speaking, we will implement a small platform using either \textit{Apache 2.4}\footnote{\url{https://httpd.apache.org/}} or \textit{nginx}\footnote{\url{http://nginx.org/}} as a web 
%   server. As a database, we can either use \textit{MySQL}\footnote{\url{http://www.mysql.com/}} or \textit{PostgreSQL}\footnote{\url{http://www.postgresql.org/}} which are relational databases. We own a small \textit{Raspberry Pi}\footnote{\url{https://www.raspberrypi.org/}} 
%   in which we can develop this platform. It will contain several things:
%   \begin{itemize}
%    \item An authentication system, which is part of the attack, an attacker wants to forge request on the behalf of an user, and he or she has to be connected.
%    \item A vulnerable service: a GET or POST service which updates data about the user (password for instance).
%    \item Some protected services with different protection methods:
%    \begin{itemize}
%     \item CSRF tokens known from the server and the client ;
%     \item Signed tokens with a private secret (only known by the server), this is a simple way to avoid saving and storing tokens but this method has several drawbacks like ``golden'' token (if the token is stolen, it can be reused on the user's behalf). This protection method will use what we've learnt from the symmetric encryption and hashing lectures of this course ;
%     \item Confirmation pop-up ;
%     \item Session information in URL in order to ``randomize'' the URL request as much as possible ;
%     \item \textit{HTTP-REFERER} check but since this value comes from the client (which has been tricked), it can't be trusted.
%    \end{itemize}
%   \end{itemize}
%   
%   According to the programming language, we don't have any preferences. Python seems to be very easy to 
%   compute with libraries like \textit{Flask}\footnote{\url{http://flask.pocoo.org/}} or 
%   \textit{Django}\footnote{\url{https://www.djangoproject.com/}} (but we won't use Django because it 
%   already provides tools against CSRF attack). We can also build a small website with PHP from 
%   scratch. This language provides lots of database integration tools\footnote{PHP provides classes as \textit{PDO} (\url{http://php.net/manual/en/book.pdo.php}) 
%   which protects from SQL injections thanks to prepared statements but this is not the point of this seminar.} 
%   and session management.
%   
%   Finally, in order to build reports and presentations (beamer), we will use \LaTeX{} and we will utilize a version control system to 
%   version and backup our code (\textit{Git}).
  

  \section{Google, vulnerable to CSRF in 2007}
  
  % TODO put in context
  
  \subsection{Contact list spoofing}
  % TODO
  \subsubsection{Description}
  % TODO
  \subsubsection{Exploitation}
  % TODO
  \subsubsection{Impact}
  % TODO
  
  
  \subsection{Email filter hijacking}
  % TODO
  \subsubsection{Description}
  % TODO
  \subsubsection{Exploitation}
  % TODO
  \subsubsection{Impact}
  % TODO
  
  \subsection{Comparison}
  % TODO
  
  
  
  \section{Conclusion}
  % TODO conclusion

  
  % TODO all \nocite{}
  
  
  
% appendix
\newpage
\appendix
\chapter{Web application's screenshots} \label{app:screenshots}
% TODO

\chapter{Practical documentation} \label{app:practical_documentation}
% TODO
  
  
% bibliography
\bibliographystyle{plainurl}
\bibliography{references}

\end{document}